\documentclass[aspectratio=169]{beamer}\usepackage[]{graphicx}\usepackage[]{xcolor}
% maxwidth is the original width if it is less than linewidth
% otherwise use linewidth (to make sure the graphics do not exceed the margin)
\makeatletter
\def\maxwidth{ %
  \ifdim\Gin@nat@width>\linewidth
    \linewidth
  \else
    \Gin@nat@width
  \fi
}
\makeatother

\definecolor{fgcolor}{rgb}{0.345, 0.345, 0.345}
\newcommand{\hlnum}[1]{\textcolor[rgb]{0.686,0.059,0.569}{#1}}%
\newcommand{\hlsng}[1]{\textcolor[rgb]{0.192,0.494,0.8}{#1}}%
\newcommand{\hlcom}[1]{\textcolor[rgb]{0.678,0.584,0.686}{\textit{#1}}}%
\newcommand{\hlopt}[1]{\textcolor[rgb]{0,0,0}{#1}}%
\newcommand{\hldef}[1]{\textcolor[rgb]{0.345,0.345,0.345}{#1}}%
\newcommand{\hlkwa}[1]{\textcolor[rgb]{0.161,0.373,0.58}{\textbf{#1}}}%
\newcommand{\hlkwb}[1]{\textcolor[rgb]{0.69,0.353,0.396}{#1}}%
\newcommand{\hlkwc}[1]{\textcolor[rgb]{0.333,0.667,0.333}{#1}}%
\newcommand{\hlkwd}[1]{\textcolor[rgb]{0.737,0.353,0.396}{\textbf{#1}}}%
\let\hlipl\hlkwb

\usepackage{framed}
\makeatletter
\newenvironment{kframe}{%
 \def\at@end@of@kframe{}%
 \ifinner\ifhmode%
  \def\at@end@of@kframe{\end{minipage}}%
  \begin{minipage}{\columnwidth}%
 \fi\fi%
 \def\FrameCommand##1{\hskip\@totalleftmargin \hskip-\fboxsep
 \colorbox{shadecolor}{##1}\hskip-\fboxsep
     % There is no \\@totalrightmargin, so:
     \hskip-\linewidth \hskip-\@totalleftmargin \hskip\columnwidth}%
 \MakeFramed {\advance\hsize-\width
   \@totalleftmargin\z@ \linewidth\hsize
   \@setminipage}}%
 {\par\unskip\endMakeFramed%
 \at@end@of@kframe}
\makeatother

\definecolor{shadecolor}{rgb}{.97, .97, .97}
\definecolor{messagecolor}{rgb}{0, 0, 0}
\definecolor{warningcolor}{rgb}{1, 0, 1}
\definecolor{errorcolor}{rgb}{1, 0, 0}
\newenvironment{knitrout}{}{} % an empty environment to be redefined in TeX

\usepackage{alltt}

% Set lecture number for later use


% Part common to all the lectures
\subtitle{MATH 8xyz -- Lecture 13}
\author{\texorpdfstring{Julien Arino\newline Department of Mathematics @ University of Manitoba \newline Maud Menten Institute @ PIMS\newline\url{julien.arino@umanitoba.ca}}{Julien Arino}}
\date{Winter 20XX}

% Title of the lecture
\title{Backward bifurcations}



\input{slides-setup-whiteBG.tex}

\IfFileExists{upquote.sty}{\usepackage{upquote}}{}
\begin{document}
%%%%%%%%%%%%%%%%%%%%%%%%%%%%%%%%%
%%%%%%%%%%%%%%%%%%%%%%%%%%%%%%%%%
%% TITLE AND OUTLINE
%%%%%%%%%%%%%%%%%%%%%%%%%%%%%%%%%
%%%%%%%%%%%%%%%%%%%%%%%%%%%%%%%%%
\titlepagewithfigure{FIGS-slides-admin/Gemini_Generated_Image_4oxcef4oxcef4oxc.jpeg}
\outlinepage{FIGS-slides-admin/Gemini_Generated_Image_tzvf9ztzvf9ztzvf.jpeg}


%%%%%%%%%%%%%%%%%%%%%%%
%%%%%%%%%%%%%%%%%%%%%%%
\subsection{A better vaccination model?}
\newSubSectionSlide{FIGS-slides-admin/Gemini_Generated_Image_vqpscpvqpscpvqps.jpeg}

\maxFrameImage{FIGS/ArinoMccluskeyPvdD.png}
\nocite{ArinoMccluskeyVdD2003}

\begin{frame}{SLIRS with vaccination}
\centering
  \def\skip{*2.5}
  \begin{tikzpicture}[scale=1.5, transform shape]
    %% Regular nodes
    \node [circle, fill=green!50, text=black] at (0,0) (S) {$S$};
    \node [circle, fill=red!90, text=black] at (1\skip,0) (I) {$I$};
    \node [circle, fill=blue!50, text=black] at (2\skip,0) (R) {$R$};
    \node [circle, fill=red!50, text=black] at (0.5\skip,-1\skip) (V) {$V$};
    %% Fake nodes for arrows
    \node [left=0.75cm of S] (birth) {};
    \node [below=0.5cm of S] (dS) {};
    \node [below=0.5cm of I] (dI) {};
    \node [below=0.5cm of R] (dR) {};
    \node [below=0.5cm of V] (dV) {};
    %% Flows (demography)
    \path [line, very thick] (birth) to node [midway, above] (TextNode) {$b$} (S);
    \path [line, very thick] (S) to node [midway, left] (TextNode) {$dS$} (dS);
    \path [line, very thick] (I) to node [midway, right] (TextNode) {$dI$} (dI);
    \path [line, very thick] (R) to node [midway, left] (TextNode) {$dR$} (dR);
    \path [line, very thick] (V) to node [midway, left] (TextNode) {$dV$} (dV);
    %% Flows
    \path [line, very thick] (S) to node [midway, above] (TextNode) {$\beta SI/N$} (I);
    \path [line, very thick, bend left=10] (S) to node [near start, right] (TextNode) {$\phi S$} (V);
    \path [line, very thick, bend left=10] (V) to node [near start, left] (TextNode) {$\varphi V$} (S);
    \path [line, very thick] (V) to node [midway, above, sloped] (TextNode) {$\sigma\beta VI/N$} (I);
    \path [line, very thick] (I) to node [midway, above] (TextNode) {$\gamma I$} (R);
    \draw [>=latex,->, thick, rounded corners] (R) -- (2\skip,0.75) -- (0,0.75) node[midway,above,sloped] {$\nu R$} -- (S);
  \end{tikzpicture}    
\end{frame}

\begin{frame}{The usual situation}
\centering
\includegraphics[width=0.7\textwidth]{FIGS/SIRV_bif_forward}
\end{frame}

\begin{frame}{What can happen with vaccination -- Backward bifurcation}
\centering
\includegraphics[width=0.7\textwidth]{FIGS/SIRV_bif_backward}
\end{frame}

% % Convert the file to R code. Only include if including R code in the document,
% % otherwise it will generate a meaningless file with just R and knitr options.
% <<convert-Rnw-to-R,warning=FALSE,message=FALSE,echo=FALSE,results='hide'>>=
% rmd_chunks_to_r_temp()
% @


\begin{frame}[allowframebreaks]{Bibliography}
\bibliographystyle{plain}
\bibliography{local-bibliography}
\end{frame}

\end{document}
