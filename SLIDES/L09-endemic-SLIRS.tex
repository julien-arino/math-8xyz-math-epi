\documentclass[aspectratio=169]{beamer}\usepackage[]{graphicx}\usepackage[]{xcolor}
% maxwidth is the original width if it is less than linewidth
% otherwise use linewidth (to make sure the graphics do not exceed the margin)
\makeatletter
\def\maxwidth{ %
  \ifdim\Gin@nat@width>\linewidth
    \linewidth
  \else
    \Gin@nat@width
  \fi
}
\makeatother

\definecolor{fgcolor}{rgb}{0.345, 0.345, 0.345}
\newcommand{\hlnum}[1]{\textcolor[rgb]{0.686,0.059,0.569}{#1}}%
\newcommand{\hlsng}[1]{\textcolor[rgb]{0.192,0.494,0.8}{#1}}%
\newcommand{\hlcom}[1]{\textcolor[rgb]{0.678,0.584,0.686}{\textit{#1}}}%
\newcommand{\hlopt}[1]{\textcolor[rgb]{0,0,0}{#1}}%
\newcommand{\hldef}[1]{\textcolor[rgb]{0.345,0.345,0.345}{#1}}%
\newcommand{\hlkwa}[1]{\textcolor[rgb]{0.161,0.373,0.58}{\textbf{#1}}}%
\newcommand{\hlkwb}[1]{\textcolor[rgb]{0.69,0.353,0.396}{#1}}%
\newcommand{\hlkwc}[1]{\textcolor[rgb]{0.333,0.667,0.333}{#1}}%
\newcommand{\hlkwd}[1]{\textcolor[rgb]{0.737,0.353,0.396}{\textbf{#1}}}%
\let\hlipl\hlkwb

\usepackage{framed}
\makeatletter
\newenvironment{kframe}{%
 \def\at@end@of@kframe{}%
 \ifinner\ifhmode%
  \def\at@end@of@kframe{\end{minipage}}%
  \begin{minipage}{\columnwidth}%
 \fi\fi%
 \def\FrameCommand##1{\hskip\@totalleftmargin \hskip-\fboxsep
 \colorbox{shadecolor}{##1}\hskip-\fboxsep
     % There is no \\@totalrightmargin, so:
     \hskip-\linewidth \hskip-\@totalleftmargin \hskip\columnwidth}%
 \MakeFramed {\advance\hsize-\width
   \@totalleftmargin\z@ \linewidth\hsize
   \@setminipage}}%
 {\par\unskip\endMakeFramed%
 \at@end@of@kframe}
\makeatother

\definecolor{shadecolor}{rgb}{.97, .97, .97}
\definecolor{messagecolor}{rgb}{0, 0, 0}
\definecolor{warningcolor}{rgb}{1, 0, 1}
\definecolor{errorcolor}{rgb}{1, 0, 0}
\newenvironment{knitrout}{}{} % an empty environment to be redefined in TeX

\usepackage{alltt}

% Set lecture number for later use


% Part common to all the lectures
\subtitle{MATH 8xyz -- Lecture 09}
\author{\texorpdfstring{Julien Arino\newline Department of Mathematics @ University of Manitoba \newline Maud Menten Institute @ PIMS\newline\url{julien.arino@umanitoba.ca}}{Julien Arino}}
\date{Winter 20XX}

% Title of the lecture
\title{The endemic SLIRS model}



\input{slides-setup-whiteBG.tex}

\IfFileExists{upquote.sty}{\usepackage{upquote}}{}
\begin{document}
%%%%%%%%%%%%%%%%%%%%%%%%%%%%%%%%%
%%%%%%%%%%%%%%%%%%%%%%%%%%%%%%%%%
%% TITLE AND OUTLINE
%%%%%%%%%%%%%%%%%%%%%%%%%%%%%%%%%
%%%%%%%%%%%%%%%%%%%%%%%%%%%%%%%%%
\titlepagewithfigure{FIGS-slides-admin/Gemini_Generated_Image_4oxcef4oxcef4oxc.jpeg}
\outlinepage{FIGS-slides-admin/Gemini_Generated_Image_tzvf9ztzvf9ztzvf.jpeg}


%%%%%%%%%%%%%%
%%%%%%%%%%%%%%
\subsection{SLIRS model with constant population}
\newSubSectionSlide{FIGS-slides-admin/Gemini_Generated_Image_vqpscpvqpscpvqps.jpeg}

\frame{\frametitle{Incubation periods}
\begin{itemize}
\item SIS and SIR: progression from S to I is instantaneous
\vfill
\item Several incubation periods:
\end{itemize}
\begin{center}
\begin{tabular}{l|l}
Disease & Incubation period \\
\hline
Yersinia Pestis & 2-6 days \\
Ebola haemorrhagic fever (HF) & 2-21 days \\
Marburg HF & 5-10 days \\
Lassa fever & 1-3 weeks \\
Tse-tse & weeks--months \\
HIV/AIDS & months--years
\end{tabular}
\end{center}
}

\frame{\frametitle{Hypotheses}
\begin{itemize}
\item There is demography
\item New individuals are born at a constant rate $b$
\item
There is no vertical transmssion: all ``newborns'' are susceptible
\item The disease is non lethal, it causes no additional mortality
\item New infections occur at the rate $f(S,I,N)$
\item There is a period of incubation for the disease
\item There is a period of time after recovery during which the disease confers immunity to reinfection (immune period)
\end{itemize} 
}

\frame{\frametitle{SLIRS}
\begin{minipage}{0.2\textwidth}
  \def\skip{*-1.5}
  \begin{tikzpicture}[scale=1, transform shape]
    %% Regular nodes
    \node [circle, fill=green!50, text=black] at (0,0) (S) {$S$};
    \node [circle, fill=red!50, text=black] at (0,1\skip) (L) {$L$};
    \node [circle, fill=red!90, text=black] at (0,2\skip) (I) {$I$};
    \node [circle, fill=blue!50, text=black] at (0,3\skip) (R) {$R$};
    %% Fake nodes for arrows
    \node [above=0.75cm of S] (birth) {};
    \node [left=0.5cm of S] (dS) {};
    \node [left=0.5cm of L] (dL) {};
    \node [left=0.5cm of I] (dI) {};
    \node [left=0.5cm of R] (dR) {};
    %% Flows (demography)
    \path [line, very thick] (birth) to node [midway, left] (TextNode) {$b$} (S);
    \path [line, very thick] (S) to node [midway, above] (TextNode) {$dS$} (dS);
    \path [line, very thick] (L) to node [midway, below] (TextNode) {$dL$} (dL);
    \path [line, very thick] (I) to node [midway, below] (TextNode) {$dI$} (dI);
    \path [line, very thick] (R) to node [midway, below] (TextNode) {$dR$} (dR);
    %% Flows (demography)
    \path [line, very thick] (S) to node [midway, left] (TextNode) {$f(S,I,N)$} (L);
    \path [line, very thick] (L) to node [midway, left] (TextNode) {$\varepsilon L$} (I);
    \path [line, very thick] (I) to node [midway, left] (TextNode) {$\gamma I$} (R);
    \draw [>=latex,->, thick, rounded corners] (R) -- (0.75,3\skip) -- (0.75,0) node[midway,above,sloped] {$\nu R$} -- (S);
  \end{tikzpicture}    
\end{minipage}
\quad
\begin{minipage}{0.7\textwidth}
The model is as follows:
\begin{subequations}\label{sys:SLIRS}
\begin{align}
S\pprime &= b+\nu R-dS-f(S,I,N) \label{sys:SLIRS_dS} \\
L\pprime &= f(S,I,N) -(d+\varepsilon)L \label{sys:SLIRS_dL} \\
I\pprime &= \varepsilon L -(d+\gamma)I \label{sys:SLIRS_dI} \\
R\pprime &= \gamma I-(d+\nu)R \label{sys:SLIRS_dR} 
\end{align}
\end{subequations}
\vfill
Meaning of the parameters:
\begin{itemize}
\item $1/\varepsilon$ average duration of the incubation period
\item $1/\gamma$ average duration of infectious period
\item $1/\nu$ average duration of immune period
\end{itemize}
\end{minipage}
}




\frame{\frametitle{Example of the SLIRS model \eqref{sys:SLIRS}}
Variation of the infected variables in \eqref{sys:SLIRS} are described by
\begin{align*}
L\pprime &= f(S,I,N)-(\varepsilon+d) L \\
I\pprime &= \varepsilon L -(d+\gamma) I
\end{align*}
Write
\begin{equation}
\I\pprime = 
\left(
\begin{matrix}
L \\
I
\end{matrix}
\right)'
=\left(
\begin{matrix}
f(S,I,N) \\
0
\end{matrix}
\right)
-
\left(
\begin{matrix}
(\varepsilon+d) L \\
(d+\gamma) I-\varepsilon L
\end{matrix}
\right)=:\mathcal{F}-\mathcal{V}
\end{equation}
}


\frame{
Denote
\[
f_L^{\,\star}:=
\left.\frac{\partial}{\partial L}f
\right|_{(S,I,R)=\bE_0}
\quad\quad 
f_I^{\,\star}:=
\left.\frac{\partial}{\partial I}f
\right|_{(S,I,R)=\bE_0}
\]
the values of the partials of the incidence function at the DFE $\bE_0$
\vfill
Compute the Jacobian matrices of vectors $\F$ and $\V$ at the DFE $\bE_0$
\begin{equation}
F=\left(
\begin{matrix}
f_L^{\,\star} & f_I^{\,\star} \\
0 & 0
\end{matrix}
\right)
\quad\text{and}\quad
V=\left(
\begin{matrix}
\varepsilon+d & 0 \\
-\varepsilon & d+\gamma
\end{matrix}
\right)
\end{equation}
}

\frame{
Thus
\[
V^{-1}=\frac{1}{(d+\varepsilon)(d+\gamma)}
\left(
\begin{matrix}
d+\gamma & 0 \\
\varepsilon & d+\varepsilon
\end{matrix}
\right)
\]
\vfill
Also, in the case $N$ is constant, $\partial f/\partial L=0$ and thus
\[
FV^{-1}=\frac{f_I^{\,\star}}
{(d+\varepsilon)(d+\gamma)}
\left(
\begin{matrix}
\varepsilon 
& d+\varepsilon  \\
0 & 0
\end{matrix}
\right)
\]
\vfill
As a consequence,
\[
\R_0=\varepsilon
\frac{f_I^{\,\star}}
{(d+\varepsilon)(d+\gamma)}
\]
}


\frame{
\begin{theorem}\label{th:R0_SEIRS}
Let
\begin{equation}\label{eq:R0_SEIRS}
\R_0=
\frac{\varepsilon f_I^{\,\star}}
{(d+\varepsilon)(d+\gamma)}
\end{equation}
Then
\begin{itemize}
\item if $\R_0<1$, the DFE is LAS
\item if $\R_0>1$, the DFE is unstable
\end{itemize}
\end{theorem}
\vfill
It is important here to stress that the result we obtain concerns the \textbf{local} asymptotic stability. We see later that even when $\R_0<1$, there can be several locally asymptotically stable equilibria
}


\frame{\frametitle{Application}
The DFE is
\[
(\bar S,\bar L,\bar I,\bar R)=(N,0,0,0)
\]
\begin{itemize}
\item Mass action incidence (frequency-dependent contacts):
 \[
f_I^{\,\star}=\beta\bar S \Rightarrow\R_0 =
\frac{\epsilon\beta N}{(\epsilon+d)(\gamma+d)} 
\]
\item Standard incidence (proportion-dependent contacts):
\[
f_I^{\,\star}=\frac{\beta\bar S}{N}
\Rightarrow\R_0 = \frac{\epsilon\beta}{(\epsilon+d)(\gamma+d)}
\]
\end{itemize}
}


\frame{\frametitle{Links between SLIRS-type models}
{ %\footnotesize
\begin{align*}
S\pprime&=b+\nu R-dS-f(S,I,N) \\
L\pprime&=f(S,I,N)-(d+\varepsilon) L \\
I\pprime&=\varepsilon L-(d+\gamma)I \\
R\pprime&=\gamma I-(d+\nu)R
\end{align*}}
\begin{center}
\begin{tabular}{c|l}
\hline
SLIR & SLIRS where $\nu=0$ \\
SLIS & Limit of SLIRS when $\nu\to\infty$ \\
SLI & SLIR where $\gamma=0$ \\
SIRS & Limit of SLIRS when $\varepsilon\to\infty$ \\
SIR & SIRS where $\nu=0$ \\
SIS & Limit of SIRS when $\nu\to\infty$ \\
& Limit SLIS when $\varepsilon\to\infty$ \\
SI & SIS where $\nu=0$ 
\end{tabular}
\end{center}
}

\frame{\frametitle{Values of $\R_0$}
$(\bar S,\bar I,\bar N)$ values of $S,I$ and $N$ at DFE. Denote $\bar f_I=\partial f/\partial I(\bar S,\bar I,\bar N)$.
\begin{center}
\begin{tabular}{c|c}
\hline
SLIRS & 
$\frac{\varepsilon\bar f_I}{(d+\varepsilon)(d+\gamma)}$ \\
SLIR & 
$\frac{\varepsilon\bar f_I}{(d+\varepsilon)(d+\gamma)}$ \\
SLIS & 
$\frac{\varepsilon\bar f_I}{(d+\varepsilon)(d+\gamma)}$ \\
SLI & 
$\frac{\varepsilon\bar f_I}{(d+\varepsilon)(d+\gamma)}$ \\
& \\
SIRS & 
$\frac{\varepsilon\bar f_I}{d+\gamma}$ \\
SIR & $\frac{\bar f_I}{d+\gamma}$ \\
SIS & $\frac{\bar f_I}{d+\gamma}$ \\
SI & $ \frac{\bar f_I}{d+\gamma}$ 
\end{tabular}
\end{center}
}


%%%%%%%%%%%%%%%%%%%%%%%
%%%%%%%%%%%%%%%%%%%%%%%
%%%%%%%%%%%%%%%%%%%%%%%
%%%%%%%%%%%%%%%%%%%%%%%
\begin{frame}[fragile]\frametitle{\textsc{Can I have this wrapped up to go?}}
To finish, we use the command \code{purl} to generate an \code{R} file (\code{course-01-introduction-math-epi.R}) in the CODE directory with all the code chunks in this \code{Rnw} file
\vfill
\begin{knitrout}
\definecolor{shadecolor}{rgb}{0.969, 0.969, 0.969}\color{fgcolor}\begin{kframe}
\begin{alltt}
\hlcom{# From https://stackoverflow.com/questions/36868287/purl-within-knit-duplicate-label-error}
\hldef{rmd_chunks_to_r_temp} \hlkwb{<-} \hlkwa{function}\hldef{(}\hlkwc{file}\hldef{)\{}
  \hldef{callr}\hlopt{::}\hlkwd{r}\hldef{(}\hlkwa{function}\hldef{(}\hlkwc{file}\hldef{,} \hlkwc{temp}\hldef{)\{}
    \hldef{out_file} \hlkwb{=} \hlkwd{sprintf}\hldef{(}\hlsng{"../CODE/%s"}\hldef{,} \hlkwd{gsub}\hldef{(}\hlsng{".Rnw"}\hldef{,} \hlsng{".R"}\hldef{, file))}
    \hldef{knitr}\hlopt{::}\hlkwd{purl}\hldef{(file,} \hlkwc{output} \hldef{= out_file,} \hlkwc{documentation} \hldef{=} \hlnum{1}\hldef{)}
  \hldef{\},} \hlkwc{args} \hldef{=} \hlkwd{list}\hldef{(file))}
\hldef{\}}
\hlkwd{rmd_chunks_to_r_temp}\hldef{(}\hlsng{"L09-endemic-SLIRS.Rnw"}\hldef{)}
\end{alltt}
\begin{verbatim}
## [1] "../CODE/L09-endemic-SLIRS.R"
\end{verbatim}
\end{kframe}
\end{knitrout}
\end{frame}


\begin{frame}[allowframebreaks]{Bibliography}
\bibliographystyle{plain}
\bibliography{local-bibliography}
\end{frame}

\end{document}
