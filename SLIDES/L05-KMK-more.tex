\documentclass[aspectratio=169]{beamer}\usepackage[]{graphicx}\usepackage[]{xcolor}
% maxwidth is the original width if it is less than linewidth
% otherwise use linewidth (to make sure the graphics do not exceed the margin)
\makeatletter
\def\maxwidth{ %
  \ifdim\Gin@nat@width>\linewidth
    \linewidth
  \else
    \Gin@nat@width
  \fi
}
\makeatother

\definecolor{fgcolor}{rgb}{0.345, 0.345, 0.345}
\newcommand{\hlnum}[1]{\textcolor[rgb]{0.686,0.059,0.569}{#1}}%
\newcommand{\hlsng}[1]{\textcolor[rgb]{0.192,0.494,0.8}{#1}}%
\newcommand{\hlcom}[1]{\textcolor[rgb]{0.678,0.584,0.686}{\textit{#1}}}%
\newcommand{\hlopt}[1]{\textcolor[rgb]{0,0,0}{#1}}%
\newcommand{\hldef}[1]{\textcolor[rgb]{0.345,0.345,0.345}{#1}}%
\newcommand{\hlkwa}[1]{\textcolor[rgb]{0.161,0.373,0.58}{\textbf{#1}}}%
\newcommand{\hlkwb}[1]{\textcolor[rgb]{0.69,0.353,0.396}{#1}}%
\newcommand{\hlkwc}[1]{\textcolor[rgb]{0.333,0.667,0.333}{#1}}%
\newcommand{\hlkwd}[1]{\textcolor[rgb]{0.737,0.353,0.396}{\textbf{#1}}}%
\let\hlipl\hlkwb

\usepackage{framed}
\makeatletter
\newenvironment{kframe}{%
 \def\at@end@of@kframe{}%
 \ifinner\ifhmode%
  \def\at@end@of@kframe{\end{minipage}}%
  \begin{minipage}{\columnwidth}%
 \fi\fi%
 \def\FrameCommand##1{\hskip\@totalleftmargin \hskip-\fboxsep
 \colorbox{shadecolor}{##1}\hskip-\fboxsep
     % There is no \\@totalrightmargin, so:
     \hskip-\linewidth \hskip-\@totalleftmargin \hskip\columnwidth}%
 \MakeFramed {\advance\hsize-\width
   \@totalleftmargin\z@ \linewidth\hsize
   \@setminipage}}%
 {\par\unskip\endMakeFramed%
 \at@end@of@kframe}
\makeatother

\definecolor{shadecolor}{rgb}{.97, .97, .97}
\definecolor{messagecolor}{rgb}{0, 0, 0}
\definecolor{warningcolor}{rgb}{1, 0, 1}
\definecolor{errorcolor}{rgb}{1, 0, 0}
\newenvironment{knitrout}{}{} % an empty environment to be redefined in TeX

\usepackage{alltt}



\input{slides-setup-whiteBG.tex}

\title{Extensions of the Kermack-McKendrick model}
\subtitle{MATH 8xyz -- Lecture 05}
\author{\texorpdfstring{Julien Arino\newline Department of Mathematics @ University of Manitoba \newline Maud Menten Institute @ PIMS\newline\url{julien.arino@umanitoba.ca}}{Julien Arino}}
\date{Winter 20XX}
\IfFileExists{upquote.sty}{\usepackage{upquote}}{}
\begin{document}
%%%%%%%%%%%%%%%%%%%%%%%%%%%%%%%%%
%%%%%%%%%%%%%%%%%%%%%%%%%%%%%%%%%
%% TITLE AND OUTLINE
%%%%%%%%%%%%%%%%%%%%%%%%%%%%%%%%%
%%%%%%%%%%%%%%%%%%%%%%%%%%%%%%%%%
\titlepagewithfigure{FIGS-slides-admin/Gemini_Generated_Image_4oxcef4oxcef4oxc.jpeg}
\outlinepage{FIGS-slides-admin/Gemini_Generated_Image_tzvf9ztzvf9ztzvf.jpeg}

%%%%%%%%%%%%%%%%%%%%%%%
%%%%%%%%%%%%%%%%%%%%%%%
\subsection{The SLIAR model}
\newSubSectionSlide{FIGS-slides-admin/Gemini_Generated_Image_vqpscpvqpscpvqps.jpeg}

\maxFrameImage{FIGS/Arino-etal-SLIAR.png}
\nocite{ArinoBrauerPvdDWatmoughWu2006}

\begin{frame}
SIR is a little too simple for many diseases:
\vfill
\begin{itemize}
\item No incubation period
\vfill
\item A lot of infectious diseases (in particular respiratory) have mild and less mild forms depending on the patient
\end{itemize}
\vfill
$\implies$ model with SIR but also L(atent) and (A)symptomatic individuals, in which I are now symptomatic individuals
\end{frame}

\begin{frame}
\centering
\resizebox{\textwidth}{!}{
  \begin{tikzpicture}[%transform canvas={scale=1.3},
      auto,
      cloud/.style={minimum width={width("N-1")+2pt},
      draw, 
      ellipse,
      fill=gray!20}]
    \node [cloud, fill=green!90] (S) {$S$};
    \node [cloud, right=2cm of S, fill=red!30] (L) {$L$};
    \node [cloud, above right=of L, fill=red!90] (I) {$I$};
    \node [cloud, below right=of L, fill=red!60] (A) {$A$};
    \node [cloud, below right=of I, fill=blue!90] (R) {$R$};
    \node [cloud, right=1.5cm of I, draw=none, fill=none] (h1) {};
    %% Infections
    \path [line, very thick] (S) to node [midway, above] (TextNode) {$\beta S(I+\delta A)$} (L);
    \path [line, very thick] (L) to node [midway, above, sloped] (TextNode) {$p\kappa L$} (I);
    \path [line, very thick] (L) to node [midway, above, sloped] (TextNode) {$(1-p)\kappa L$} (A);
    \path [line, very thick] (I) to node [midway, above, sloped] (TextNode) {$f\alpha I$} (R);
    \path [line, very thick] (A) to node [midway, above, sloped] (TextNode) {$\eta A$} (R);
    \path [line, very thick] (I) to node [midway, above, sloped] (TextNode) {$(1-f)\alpha I$} (h1);
  \end{tikzpicture}
}
\end{frame}

\begin{frame}{Basic reproduction number \& Final size}
We find the basic reproduction number
\begin{equation}
\mathcal{R}_0=\beta
\left(
\frac{p}{\alpha}+\frac{\delta(1-p)}{\eta}
\right)S_0
=\frac{\beta\rho}{\alpha}S_0
\end{equation}
where 
\[
\rho = \alpha
\left(
\frac{p}{\alpha}+\frac{\delta(1-p)}{\eta}
\right)
\]
\vfill
The final size relation takes the form
\begin{equation}
S_0(\ln S_0-\ln S_\infty) =
\mathcal{R}_0(S_0-S_\infty)+\frac{\mathcal{R}_0I_0}{\rho}
\end{equation}
\end{frame}

\begin{frame}{Adding treatment}
\centering
\resizebox{0.8\textheight}{!}{
\def\horzskip{*2.75cm}
\def\vertskip{*1.5cm}
  \begin{tikzpicture}[%transform canvas={scale=1.3},
      auto,
      cloud/.style={minimum width={width("L\_T")+2pt},
      draw, 
      ellipse,
      fill=gray!20}]
    \node [cloud, fill=green!90] at (0,2\vertskip) (S) {$S$};
    \node [cloud, fill=red!30] at (1\horzskip,2\vertskip) (L) {$L$};
    \node [cloud, fill=red!90] at (2\horzskip,1\vertskip) (I) {$I$};
    \node [cloud, fill=red!60] at (2\horzskip,3\vertskip) (A) {$A$};
    \node [cloud, fill=blue!90] at (3\horzskip,0) (R) {$R$};
    \node [cloud, fill=green!90] at (0,-2\vertskip) (ST) {$S_T$};
    \node [cloud, fill=red!30] at (1\horzskip,-2\vertskip) (LT) {$L_T$};
    \node [cloud, fill=red!90] at (2\horzskip,-1\vertskip) (IT) {$I_T$};
    \node [cloud, fill=red!60] at (2\horzskip,-3\vertskip) (AT) {$A_T$};
    %% Flows untreated
    \path [line, very thick] (S) to node [midway, above] (TextNode) {$S\beta Q$} (L);
    \path [line, very thick] (L) to node [midway, above, sloped] (TextNode) {$p\kappa L$} (I);
    \path [line, very thick] (L) to node [midway, above, sloped] (TextNode) {$(1-p)\kappa L$} (A);
    \path [line, very thick] (I) to node [midway, above, sloped] (TextNode) {$f\alpha I$} (R);
    \path [line, very thick] (A) to node [midway, above, sloped] (TextNode) {$\eta A$} (R);
    %% Flows treated
    \path [line, very thick] (ST) to node [midway, above] (TextNode) {$\sigma_SS_T\beta Q$} (LT);
    \path [line, very thick] (LT) to node [midway, above, sloped] (TextNode) {$p\tau\kappa_T L_T$} (IT);
    \path [line, very thick] (LT) to node [midway, below, sloped] (TextNode) {$(1-p\tau)\kappa_T L_T$} (AT);
    \path [line, very thick] (IT) to node [midway, above, sloped] (TextNode) {$f_T\alpha_T I_T$} (R);
    \path [line, very thick] (AT) to node [midway, below, sloped] (TextNode) {$\eta_T A_T$} (R);
    %% Flow treatment
    \path [line, very thick, bend left=10] (LT) to node [midway, above, sloped] (TextNode) {$\theta_LL_T$} (L);
    \path [line, very thick, bend left=10] (L) to node [midway, above, sloped] (TextNode) {$\varphi_LL$} (LT);
    \path [line, very thick, bend left=10] (IT) to node [midway, above, sloped] (TextNode) {$\theta_II_T$} (I);
    \path [line, very thick, bend left=10] (I) to node [midway, above, sloped] (TextNode) {$\varphi_II$} (IT);
  \end{tikzpicture}
}
\end{frame}

\maxFrameImage{FIGS/Arino_etal-SLIAR-treatment-doses.png}
\maxFrameImage{FIGS/Arino_etal-SLIAR-treatment-cases.png}
\maxFrameImage{FIGS/Arino_etal-SLIAR-conclusions.png}

%%%%%%%%%%%%%%%%%%%%%%%
%%%%%%%%%%%%%%%%%%%%%%%
%%%%%%%%%%%%%%%%%%%%%%%
%%%%%%%%%%%%%%%%%%%%%%%
\subsection{Computing the final size more efficiently}
\newSubSectionSlide{FIGS-slides-admin/Gemini_Generated_Image_vqpscpvqpscpvqps.jpeg}

\maxFrameImage{FIGS/Arino-etal-final-size.png}
\nocite{ArinoBrauerDriesscheWatmoughWu2007}

\begin{frame}{A method for computing $\mathcal{R}_0$ in epidemic models}
\bbullet This method is not universal! It works in a relatively large class of models, but not everywhere
\vfill
\bbullet If it doesn't work, the next generation matrix method does work, \textbf{but} should be considered only for obtaining the reproduction number, not to deduce LAS
\vfill
\bbullet Here, I change the notation in the paper, for convenience
\end{frame}

\begin{frame}{Standard form of the system}
Suppose system can be written in the form
\begin{subequations}\label{sys:SIR_general}
\begin{align}
\bS\pprime &= \mathbf{b}(\bS,\bI,\bR)-\bD\bS\beta(\bS,\bI,\bR)\bh\bI \label{sys:SIR_general_dS} \\
\bI\pprime &= \bPi\bD\bS\beta(\bS,\bI,\bR)\bh\bI-\mathbf{V}\bI \label{sys:SIR_general_dI} \\
\bR\pprime &= \mathbf{f}(\bS,\bI,\bR)+\mathbf{W}\bI \label{sys:SIR_general_dR}
\end{align}
\end{subequations}
\vfill
where $\bS\in\IR^m$, $\bI\in\IR^n$ and $\bR\in\IR^k$ are susceptible, infected and removed compartments, respectively
\vfill
IC are $\geq 0$ with at least one of the components of $\bI(0)$ positive
\end{frame}  


\begin{frame}
\begin{equation}\tag{\ref{sys:SIR_general_dS}}
\bS\pprime = \mathbf{b}(\bS,\bI,\bR)-\bD\bS\beta(\bS,\bI,\bR)\bh\bI
\end{equation}
\begin{itemize}
\item $\mathbf{b}:\IR_+^m\times\IR_+^n\times\IR_+^k\to\IR^m$ continuous function encoding recruitment and death of uninfected individuals
\item $\bD\in\IR^{m\times m}$ diagonal with diagonal entries $\sigma_i>0$ the relative susceptibilities of susceptible compartments, with convention that $\sigma_1=1$
\item Scalar valued function $\beta:\IR_+^m\times\IR_+^n\times\IR_+^k\to\IR_+$ represents infectivity, with, e.g., $\beta(\bS,\bI,\bR)=\beta$ for mass action
\item $\bh\in\IR^{n}$ row vector of relative horizontal transmissions
\end{itemize}
\end{frame}  


\begin{frame}
\begin{equation}\tag{\ref{sys:SIR_general_dI}}
\bI\pprime = \bPi\bD\bS\beta(\bS,\bI,\bR)\bh\bI-\mathbf{V}\bI
\end{equation}
\begin{itemize}
\item $\bPi\in\IR^{n\times m}$ has $(i,j)$ entry the fraction of individuals in $j^{\textrm{th}}$ susceptible compartment that enter $i^{\textrm{th}}$ infected compartment upon infection
\item $\bD\in\IR^{m\times m}$ diagonal with diagonal entries $\sigma_i>0$ the relative susceptibilities of susceptible compartments, with convention that $\sigma_1=1$
\item Scalar valued function $\beta:\IR_+^m\times\IR_+^n\times\IR_+^k\to\IR_+$ represents infectivity, with, e.g., $\beta(\bS,\bI,\bR)=\beta$ for mass action
\item $\bh\in\IR^{n}$ row vector of relative horizontal transmissions
\item $\mathbf{V}\in\IR^{n\times n}$ describes transitions between infected states and removals from these states due to recovery or death
\end{itemize}
\end{frame}  


\begin{frame}
\begin{equation}\tag{\ref{sys:SIR_general_dR}}
\bR\pprime = \mathbf{f}(\bS,\bI,\bR)+\mathbf{W}\bI
\end{equation}
\begin{itemize}
\item $\mathbf{f}:\IR_+^m\times\IR_+^n\times\IR_+^k\to \IR^k$ continuous function encoding flows into and out of removed compartments because of immunisation or similar processes
\item $\mathbf{W}\in\IR^{k\times n}$ has $(i,j)$ entry the rate at which individuals in the $j^{\textrm{th}}$ infected compartment move into the $i^{\textrm{th}}$ removed compartment
\end{itemize}
\end{frame}



\begin{frame}
Suppose $\bE_0$ is a locally stable disease-free equilibrium (DFE) of the system without disease, i.e., an EP of
\begin{align*}
\bS\pprime &= \mathbf{b}(\bS,\b0,\bR) \\
\bR\pprime &= \mathbf{f}(\bS,\b0,\bR) \\
\end{align*}

\begin{theorem}
Let
\begin{equation}\label{eq:R0_final_size_method}
\mathcal{R}_0 = 
\beta(\bS_0,\b0,\bR_0)
\bh\mathbf{V}^{-1}
\bPi\bD\bS_0
\end{equation}
\begin{itemize}
\item If $\mathcal{R}_0<1$, the DFE $\bE_0$ is a locally asymptotically stable EP of \eqref{sys:SIR_general}
\item If $\mathcal{R}_0>1$, the DFE $\bE_0$ of \eqref{sys:SIR_general} is unstable
\end{itemize}
\end{theorem}
\vfill
If no demography (epidemic model), then just $\R_0$, of course
\end{frame}  

\begin{frame}{Final size relations}
Assume no demography, then system should be writeable as
\begin{subequations}
\begin{align}
\bS\pprime &= -\bD\bS\beta(\bS,\bI,\bR)\bh\bI  \label{sys:SIR_epi_dS} \\
\bI\pprime &= \bPi\bD\bS\beta(\bS,\bI,\bR)\bh\bI-\mathbf{V}\bI 
\label{sys:SIR_epi_dI} \\
\bR\pprime &= \mathbf{W}\bI
\label{sys:SIR_epi_dR} 
\end{align}
\end{subequations}
\vfill
For $w(t)\in\IR_+^n$ continuous, define
$$
w_\infty = \lim_{t\to\infty}w(t)\quad\text{and}\quad
\hat{w}=\int_0^\infty w(t)\ dt
$$
\end{frame}  

\begin{frame}
Define the row vector 
\[
\IR^m\ni\bGamma
=(\Gamma_1,\ldots,\Gamma_m)=\beta(\bS_0,\b0,\bR_0)\bh\mathbf{V}^{-1}\bPi\bD
\]
then 
\[
\mathcal{R}_0=\bGamma\bS(0)
\]
\end{frame}  

\begin{frame}
Suppose incidence is mass action, i.e., $\beta(\bS,\bI,\bR)=\beta$ and $m>1$
\vfill
Then for $i=1,\ldots,m$, express $\bS_i(\infty)$ as a function of $\bS_1(\infty)$ using
$$
\bS_i(\infty)  = 
\bS_i(0) \left(
\frac{\bS_1(\infty)}{\bS_1(0)}
\right)^{\sigma_i/\sigma_1}
$$
then substitute into 
\begin{align*}
\frac{1}{\sigma_i}
\ln\left(\frac{\bS_i(0)}{\bS_i(\infty)}\right)
&=
\bGamma\bD^{-1}\left(\bS(0)-\bS(\infty)\right)
+\beta\bh\mathbf{V}^{-1}\bI(0) \\
&= 
\frac{1}{\sigma_1}
\ln\left(\frac{\bS_1(0)}{\bS_1(\infty)}\right)
\end{align*}
which is a final size relation for the general system when $\bS_i(0)>0$
\end{frame}  


\begin{frame}
If incidence is mass action and $m=1$ (only one susceptible compartment), reduces to the KMK form
\begin{equation}
\label{eq:final_size_m1}
\ln\left(
\frac{S_0}{S_\infty}
\right)
=\frac{\mathcal{R}_0}{S_0}
(S_0-S_\infty)+\beta\bh\mathbf{V}^{-1}\bI_0
\end{equation}
\end{frame}  


\begin{frame}
In the case of more general incidence functions, the final size relations are inequalities of the form, for $i=1,\ldots,m$,
$$
\ln\left(\frac{\bS_i(0)}{\bS_i(\infty)}\right)
\geq
\sigma_i\bGamma\bD^{-1}\left(\bS(0)-\bS(\infty)\right)
+\sigma_i\beta(K)\bh\mathbf{V}^{-1}\bI(0)
$$
where $K$ is the initial total population
\end{frame}  

%%%%%%%%%%%%%%%%%%%%
%%%%%%%%%%%%%%%%%%%%
\subsection{A variation on the SLIAR model}
\newSubSectionSlide{FIGS-slides-admin/Gemini_Generated_Image_vqpscpvqpscpvqps.jpeg}

\begin{frame}{The SLIAR model}
\bbullet Paper we have already seen: Arino, Brauer, PvdD, Watmough \& Wu. \href{http://dx.doi.org/10.1098/rsif.2006.0112}{Simple models for containment of a pandemic}, \emph{Journal of the Royal Society Interface} (2006)
\vfill
\bbullet However, suppose additionally that $L$ are also infectious
\end{frame}  

\begin{frame}
\centering
\includegraphics[width=\textwidth]{FIGS/SLIAR_infectiousL}
\end{frame}  


\begin{frame}
Here, $\bS=S$, $\bI=(L,I,A)^T$ and $\bR=R$, so $m=1$, $n=3$ and 
$$
\bh=[\varepsilon\; 1\; \delta],
\quad
\bD=1,
\quad 
\bPi
=\begin{pmatrix}
1 \\ 0 \\0
\end{pmatrix}
\quad\text{and}\quad
\mathbf{V}=
\begin{pmatrix}
\kappa & 0 & 0 \\
-p\kappa & \alpha & 0 \\
-(1-p)\kappa & 0 & \eta
\end{pmatrix}
$$
Incidence is mass action so $\beta(\bE_0)=\beta$ and thus
\begin{align*}
\mathcal{R}_0
&=
\beta\bh\mathbf{V}^{-1}\bPi\bD\bS_0 \\
&=
\beta\;
[\varepsilon\; 1\; \delta]
\begin{pmatrix}
1/\kappa & 0 & 0 \\
p/\alpha & 1/\alpha & 0 \\
(1-p)/\eta & 0 & 1/\eta
\end{pmatrix}
\begin{pmatrix}
1 \\ 0 \\0
\end{pmatrix}
S_0 \\
&=
\beta S_0\left(
\frac{\varepsilon}{\kappa}
+\frac{p}{\alpha}
+\frac{\delta(1-p)}{\eta}
\right)
\end{align*}
\end{frame}  

\begin{frame}
For final size, since $m=1$, we can use $\eqref{eq:final_size_m1}$:
\[
\ln\left(
\frac{S_0}{S_\infty}
\right)
=\frac{\mathcal{R}_0}{S_0}
(S_0-S_\infty)+\beta\bh\mathbf{V}^{-1}\bI_0
\]
\vfill
Suppose $\bI_0=(0,I_0,0)$, then
\[
\ln\left(
\frac{S_0}{S_\infty}
\right)
=\mathcal{R}_0\frac{S_0-S_\infty}{S_0}
+\frac{\beta}{\alpha}I_0
\]
\vfill
If $\bI_0=(L_0,I_0,A_0)$, then
\[
\ln\left(
\frac{S_0}{S_\infty}
\right)
=\mathcal{R}_0\frac{S_0-S_\infty}{S_0}
+\beta\left(
\frac{\varepsilon}{\kappa}
+\frac{p}{\alpha}
+\frac{\delta(1-p)}{\eta}
\right)L_0
+\frac{\beta\delta}{\eta}A_0
+\frac{\beta}{\alpha}I_0
\]
\end{frame}  



%%%%%%%%%%%%%%%%%%%%
%%%%%%%%%%%%%%%%%%%%
\subsection{A model with vaccination}
\newSubSectionSlide{FIGS-slides-admin/Gemini_Generated_Image_vqpscpvqpscpvqps.jpeg}

\begin{frame}{A model with vaccination}
\centering
\begin{tikzpicture}[auto, %node distance = 2cm, auto,
	cloud/.style={minimum width={width("N-1")+2pt},
		draw, ellipse,fill=gray!20}]
    \node [cloud, fill=green!90] (SU) {$S_U$};
    \node [cloud, below=2cmof SU, fill=green!90] (SV) {$S_V$};
    \node [cloud, right=3cm of SU, fill=red!30] (LU) {$L_U$};
    \node [cloud, right=3cm of SV, fill=red!30] (LV) {$L_V$};
	\node [cloud, right=of LU, fill=red!90] (IU) {$I_U$};
	\node [cloud, right=of LV, fill=red!90] (IV) {$I_V$};
	\node [cloud, below right=of IU, fill=blue!90] (R) {$R$};
    \node [cloud, above right=2cm of IU, draw=none, fill=none] (h1) {};
    \node [cloud, below right=2cm of IV, draw=none, fill=none] (h2) {};
	%% Infections
	\path [line, very thick] (SU) to node [midway, above] (TextNode) {$\beta S_U(I_U+\sigma_II_V)$} (LU);
	\path [line, very thick] (SV) to node [midway, above] (TextNode) {$\sigma_S\beta S_V(I_U+\sigma_II_V)$} (LV);
	\path [line, very thick] (LU) to node [midway, above, sloped] (TextNode) {$\kappa_U L_U$} (IU);
	\path [line, very thick] (LV) to node [midway, above, sloped] (TextNode) {$\kappa_V L_V$} (IV);
	\path [line, very thick] (IU) to node [midway, above, sloped] (TextNode) {$f_U\alpha_U I_U$} (R);
	\path [line, very thick] (IV) to node [midway, above, sloped] (TextNode) {$f_V\alpha_V I_V$} (R);
    \path [line, very thick] (IU) to node [midway, above, sloped] (TextNode) {$(1-f_U)\alpha_U I_U$} (h1);
    \path [line, very thick] (IV) to node [midway, below, sloped] (TextNode) {$(1-f_V)\alpha_V I_V$} (h2);
\end{tikzpicture}
\end{frame}  


\begin{frame}{A model with vaccination}
Fraction $\gamma$ of $S_0$ are vaccinated before the epidemic; vaccination reduces probability and duration of infection, infectiousness and reduces mortality
\begin{subequations}
\begin{align}
S_U\pprime &= -\beta S_U[I_U+\sigma_II_V] \\
S_V\pprime &= -\sigma_S\beta S_V[I_U+\sigma_II_V] \\
L_U\pprime &= \beta S_U[I_U+\sigma_II_V]-\kappa_UL_U\\
L_V\pprime &= \sigma_S\beta S_V[I_U+\sigma_II_V]-\kappa_VL_V \\
I_U\pprime &= \kappa_UL_U-\alpha_UI_U \\
I_V\pprime &= \kappa_VL_V-\alpha_VI_V \\
R\pprime &= f_U\alpha_UI_I+f_V\alpha_VI_V
\end{align}
\end{subequations}
with $S_U(0)=(1-\gamma)S_0$ and $S_V(0)=\gamma S_0$
\end{frame}  


\begin{frame}
Here, $m=2$, $n=4$,
\[
\bh = [0\;0\;1\;\sigma_I],\quad
\bD=\begin{pmatrix}
1 & 0 \\ 0 & \sigma_S
\end{pmatrix},\quad
\bPi=
\begin{pmatrix}
1 & 0 \\ 0 & 1 \\ 0 & 0 \\ 0 & 0
\end{pmatrix}
\]
and
\[
\mathbf{V}=
\begin{pmatrix}
\kappa_U & 0 & 0 & 0 \\
0 & \kappa_V & 0 & 0 \\
-\kappa_U & 0 & \alpha_U & 0 \\
0 & -\kappa_V & 0 & \alpha_V
\end{pmatrix}
\]
\end{frame}

\begin{frame}
So
\[
\bGamma=\left[
\frac{\beta}{\alpha_U}\; \frac{\sigma_I\sigma_S\beta}{\alpha_V}
\right],
\quad
\mathcal{R}_c = S_0\beta\left(
\frac{1-\gamma}{\alpha_U}+\frac{\sigma_I\sigma_S\gamma}{\alpha_V}
\right)
\]
and the final size relation is
\begin{multline*}
\ln\left(
\frac{(1-\gamma)S_U(0)}{S_U(\infty)}
\right)
= \\ 
\frac{\beta}{\alpha_U}[(1-\gamma)S_U(0)-S_U(\infty)] \\
+\frac{\sigma_I\beta}{\alpha_V}[\gamma S_V(0)-S_V(\infty)]+\frac{\beta}{\alpha_U}I_0 \\
\end{multline*}
\[
S_V(\infty) = \gamma S_U(0)\left(
\frac{S_U(\infty)}{(1-\gamma)S_0}
\right)^{\sigma_S}
\]
\end{frame}

%%%%%%%%%%%%%%%%%%%%
%%%%%%%%%%%%%%%%%%%%
\subsection{Antiviral resistance}
\newSubSectionSlide{FIGS-slides-admin/Gemini_Generated_Image_vqpscpvqpscpvqps.jpeg}

\maxFrameImage{FIGS/ArinoBowmanMoghadas.png}
\nocite{ArinoBowmanMoghadas2009}

\begin{frame}{Adapting treatment to counter emergence of resistance}
This work was undertaken at the request of the Public Health Agency of Canada during the pandemic preparadness phase prior to the 2009 p-H1N1 pandemic
\vfill
Problem: we have antivirals to use against influenza, either prophylactically or curatively. Using these antivirals may promote the emergence of antiviral-resistant strains. How do we minimise this risk?
\end{frame}


\begin{frame}
\centering
\resizebox{\textwidth}{!}{
  \begin{tikzpicture}[%transform canvas={scale=1.3},
      auto,
      cloud/.style={minimum width={width("N-1")+2pt},
      draw, 
      ellipse,
      fill=gray!20}]
    %% S
    \node [cloud, fill=green!90] at (0,0) (S) {$S$};
    %% Untreated, resistant
    \node [cloud, fill=red!90] at (3,1) (A_r) {$A_r$};
    \node [cloud, fill=red!60] at (3,-1) (I_rU) {$I_{rU}$};
    \node [cloud, fill=blue!60] at (6,0) (R_rU) {$R_{rU}$};
    %% Untreated, sensitive
    \node [cloud, fill=red!90] at (-3,1) (A) {$A$};
    \node [cloud, fill=red!60] at (-3,-1) (I_U) {$I_U$};
    \node [cloud, fill=blue!60] at (-6,0) (R_U) {$R_U$};
    %% Treated
    \node [cloud, fill=red!90] at (-1.5,-3) (I_T) {$I_T$};
    \node [cloud, fill=red!90] at (1.5,-3) (I_rT) {$I_{rT}$};
    \node [cloud, fill=red!90] at (0,-6) (I_Tr) {$I_{Tr}$};
    %% Resistance
    \node [cloud, fill=blue!60] at (-4.5,-3) (R_T) {$R_T$};
    \node [cloud, fill=blue!60] at (4.5,-3) (R_rT) {$R_{rT}$};
    \node [cloud, fill=blue!60] at (3,-6) (R_Tr) {$R_{Tr}$};
    %%
    %% Infections
    \path [line, very thick] (S) to node [midway,above,sloped] (TextNode) {$(1-p)fS$} (A);
    \path [line, very thick] (S) to node [midway,below,sloped] (TextNode) {$(1-q)pfS$} (I_U);
    \path [line, very thick] (S) to node [midway,above,sloped] (TextNode) {$(1-p)gS$} (A_r);
    \path [line, very thick] (S) to node [midway,below,sloped] (TextNode) {$(1-q)pgS$} (I_rU);
    \path [line, very thick] (S) to node [near end,above,sloped] (TextNode) {$qpfS$} (I_T);
    \path [line, very thick] (S) to node [midway,below,sloped] (TextNode) {$qpgS$} (I_rT);
    %% Removals
    \path [line, very thick] (A) to node [midway,below,sloped] (TextNode) {$\mu_AA$} (R_U);
    \path [line, very thick] (I_U) to node [midway,below,sloped] (TextNode) {$(d_U+\mu_U)I_U$} (R_U);
    \path [line, very thick] (A_r) to node [midway,below,sloped] (TextNode) {$\mu_AA_r$} (R_rU);
    \path [line, very thick] (I_rU) to node [midway,below,sloped] (TextNode) {$(d_{rU}+\mu_U)I_{rU}$} (R_rU);
    \path [line, very thick] (I_T) to node [midway,below,sloped] (TextNode) {$(d_T+\mu_T)I_T$} (R_T);
    \path [line, very thick] (I_rT) to node [midway,below,sloped] (TextNode) {$(d_{rU}+\mu_U)I_{rT}$} (R_rT);
    \path [line, very thick] (I_Tr) to node [midway,below,sloped] (TextNode) {$(d_{Ur}+\mu_U)I_{Tr}$} (R_Tr);
    %% I_T transitions
        \path [line, very thick] (I_T) to node [midway,below,sloped] (TextNode) {$\alpha I_T$} (I_Tr);
    \path [line, very thick] (I_Tr) to node [midway,below,sloped] (TextNode) {$\gamma I_{Tr}$} (I_rT);
  \end{tikzpicture}
}
\end{frame}

%%%%%%%%%%%%%%%%%%%
%%%%%%%%%%%%%%%%%%%
\subsection{A COVID-19 model}
\newSubSectionSlide{FIGS-slides-admin/Gemini_Generated_Image_vqpscpvqpscpvqps.jpeg}

\maxFrameImage{FIGS/ArinoPortet-2020.png}
\nocite{ArinoPortet2020}

\begin{frame}
Extends the SLIAR model to take into account non-exponentially distributed stage durations (see course 02)
\end{frame}

\begin{frame}{The original model (well, almost the first one)}
\centering
\def\horzskip{*2}
\def\vertskip{*2}
\begin{tikzpicture}[auto, %node distance = 2cm, auto,
	cloud/.style={minimum width={width("N-1")+2pt},
		draw, ellipse,fill=red!20}]
	\node [cloud] (S) at (0,0) {$S$};
	\node [cloud] (L1) at (1\horzskip,0) {$L_1$};
	\node [cloud] (L2) at (2\horzskip,0) {$L_2$};
	\node [cloud,fill=blue!20] (I1) at (3\horzskip,-1\vertskip) {$I_1$};
	\node [cloud] (A1) at (3\horzskip,1\vertskip) {$A_1$};
	\node [cloud,fill=blue!20] (I2) at (4\horzskip,-1\vertskip) {$I_2$};
	\node [cloud] (A2) at (4\horzskip,1\vertskip) {$A_2$};
	\node [cloud,fill=blue!20] (RI) at (5\horzskip,0) {$R_I$};
	\node [cloud] (RA) at (5\horzskip,1\vertskip) {$R_A$};
	\node [cloud,, fill=blue!20] (D) at (5\horzskip,-2\vertskip) {$D$};
	%% Infections
	\path [line, very thick] (S) to node [midway,above] (TextNode) {$\Phi S$} (L1);
	\path [line, very thick] (L1) to node [midway,above] (TextNode) {$\varepsilon L_1$} (L2);
	\path [line, very thick] (L2) to node [midway,below,sloped] (TextNode) {$(1-\pi)\varepsilon L_2$} (I1);
	\path [line, very thick] (L2) to node [midway,below,sloped] (TextNode) {$\pi\varepsilon L_2$} (A1);
	\path [line, very thick] (I1) to node [midway, above] (TextNode) {$\gamma I_1$} (I2);
	\path [line, very thick] (A1) to node [midway, above] (TextNode) {$\gamma A_1$} (A2);
	\path [line, very thick] (I2) to node [midway,above,sloped] (TextNode) {$(1-\delta)\gamma I_2$} (RI);
	\path [line, very thick] (A2) to node [midway, above] (TextNode) {$\gamma A_2$} (RA);
	\path [line, very thick] (I2) to node [midway,below,sloped] (TextNode) {$\delta\gamma I_2$} (D);
\end{tikzpicture}
\end{frame}


\begin{frame}{Reinterpreting terms}
Here $D$ stands for \emph{detected}, $U$ is \emph{undetected}
\vfill
\centering
\includegraphics[width=\textwidth]{FIGS/figure_SLDURM_base_model_with_different_epsilon_and_infectious_compartments}
\end{frame}

\begin{frame}{Working out when the first COVID-19 case occurred}
\bbullet Details of emergence and precise timeline before amplification started unknown
\vfill
\bbullet Amplification in Wuhan
\begin{itemize}
\item Cluster of pneumonia cases mostly related to the Huanan Seafood Market
\item 27 December 2019: first report to local government
\item 31 December 2019: publication
\item 8 January 2020: identification of SARS-CoV-2 as causative agent
\item $\sim$ 23 January 2020: lockdown Wuhan and Hubei province + face mask mandates
\end{itemize}
\vfill
\bbullet By 2020-01-29, virus in all provinces of mainland CHN
\end{frame}


\begin{frame}{Evidence of earlier spread}
\bbullet Report to Wuhan authorities on 27 December 2019
\vfill
\bbullet First export detections in Thailand and Japan on 13 and 16 January 2020 (with actual importations on 8 and 6 January)
\vfill
$\implies$ amplification must have been occuring for a while longer
\vfill
\bbullet France: sample taken from 42-year-old male (last foreign travel to Algeria in August 2019) who presented to ICU on 27 December 2019
\vfill
\bbullet Retrospective studies in United Kingdom and Italy also showed undetected COVID-19 cases in prepandemic period
\end{frame}

\begin{frame}{Untangling the first case issue}
\bbullet Robert, Rossman \& Jaric. Dating first cases of COVID-19. \emph{PLoS Pathogens} (2021)

Find likely timing of first case of COVID-19 in China as November 17 (95\% CI October 4)
\vfill
\bbullet Pekar, Worobey, Moshiri, Scheffler \& Wertheim. Timing the SARS-CoV-2 index case in Hubei province. \emph{Science} (2021)

Period between mid-October and mid-November 2019 is plausible interval when the first case of SARS-CoV-2 emerged in Hubei province
\vfill
Important when trying to understand global spread, so let me illustrate with the model I used, taking into account model evolution since
\end{frame}

\begin{frame}{Back-calculating the start of spread (example of China)}
Cumulative confirmed case counts in China as reported to WHO was $c=547$ cases on $t_c=\textrm{2020-01-22}$
\vfill
Let $u$ be a point in parameter space. Solve ODE numerically over $[0,t]$, with $S(0)$ the population of China, $L_1(0)=1$ and other state variables 0. This gives a solution $x(t,t_0=0,u)$
\vfill
Extracting $L_2(t,t_0=0,u)$ from this solution, obtain cumulative number of new detections as
\[
C(t) = \int_{t_0=0}^{t} p\varepsilon_2 L_2(s,t_0,u)\ ds
\]
\vfill
Let $t^\star$ be s.t. $C(t^\star)=547$; then $t_i=\textrm{2020-01-22}-t^\star$
\end{frame}

\begin{frame}
\centering
\includegraphics[width=\textwidth]{FIGS/start_time_vs_R0_and_p.png}
\end{frame}




%%%%%%%%%%%%%%%%%%%%%%%%%%
%%%%%%%%%%%%%%%%%%%%%%%%%%
%%%%%%%%%%%%%%%%%%%%%%%%%%
%%%%%%%%%%%%%%%%%%%%%%%%%%


%%%%%%%%%%%%%%%%%%%%%%%%%%
%%%%%%%%%%%%%%%%%%%%%%%%%%
%%%%%%%%%%%%%%%%%%%%%%%%%%
%%%%%%%%%%%%%%%%%%%%%%%%%%
\begin{frame}[allowframebreaks]{Bibliography}
\bibliographystyle{plain}
\bibliography{local-bibliography}
\end{frame}

\end{document}
